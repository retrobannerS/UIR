\title{Разработка веб-сервиса для запросов к базам данных \\ на естественном языке}

\taskdate{24.02.2025}

\projecttasks{
  \section{== TODO ==}
  \projecttask{\bfseries\projecttasknum}{\bfseries Аналитическая часть}{}{}{}
  % (указываются предмет и цели анализа)
  % ----------------------------------------
  \projecttask{\projectsubtasknum}
  {
    Провести анализ существующих веб-
    сервисов, позволяющих производить запросы
    к базам данных на естественном языке.
  }%
  {Список литературы, текст РСПЗ}%
  {14.03.2025}
  {
    % \signat[xsign=0pt, ysign=-10pt, scale=0.4, xdate=14pt, ydate=-22pt, date={01.01.2001}]
  }
  % ----------------------------------------
  \projecttask{\projectsubtasknum}
  {
    Провести анализ существующих моделей,
    переводящих естественный язык на язык
    запросов SQL.
  }%
  {Список литературы, текст РСПЗ}%
  {24.03.2025}
  {
    % \signat[xsign=-7pt, ysign=-7pt, scale=0.32, xdate=12pt, ydate=-17pt, date={01.01.2001}]
  }
  % ----------------------------------------
  \projecttask{\projectsubtasknum}
  {\itshape
    Оформить расширенное содержание
    пояснительной записки (РСПЗ)
  }%
  {Текст РСПЗ}%
  {28.03.2025}
  {
    % \signat[xsign=-12pt, ysign=-4pt, scale=0.25, xdate=14pt, ydate=-10pt, date={\footnotesize01.01.2001}]
  }
  % ----------------------------------------
  % ----------------------------------------
  \projecttask{\bfseries\projecttasknum}{\bfseries Теоретическая часть}{}{}{}
  % (указываются используемые и разрабатываемые модели, методы, алгоритмы)
  \projecttask{\projectsubtasknum}
  {
    Постановить задачу о проектировании
    внутренней базы данных и разработке блока
    предобработки запросов
  }%
  {Текст ПЗ}%
  {03.04.2025}
  {
    % \signat[xsign=-12pt, ysign=-4pt, scale=0.25, xdate=14pt, ydate=-10pt, date={\footnotesize01.01.2001}]
  }
  % ----------------------------------------
  \projecttask{\projectsubtasknum}
  {Спроектировать внутреннюю базу данных}%
  {Описание моделей БД, текст ПЗ}%
  {10.04.2025}
  {
    % \signat[xsign=-7pt, ysign=-7pt, scale=0.32, xdate=12pt, ydate=-17pt, date={01.01.2001}]
  }
  % ----------------------------------------
  \projecttask{\projectsubtasknum}
  {Разработать блок предобработки запросов}%
  {Текст ПЗ}%
  {17.04.2025}
  {
    % \signat[xsign=-17pt, ysign=-2pt, scale=0.20, xdate=19pt, ydate=0pt, date={\scriptsize01.01.2001}]
  }
  % ----------------------------------------
  % ----------------------------------------
  \projecttask{\bfseries\projecttasknum}{\bfseries Инженерная часть}{}{}{}
  % (указывается, что конкретно необходимо спроектировать, а также используемые для этого методы, технологии и инструментальные средства)
  \projecttask{\projectsubtasknum}
  {
    Установить требования к системе и
    интерфейсу пользователя
  }%
  {Текст ПЗ}%
  {24.04.2025}
  {
    % \signat[xsign=-10pt, ysign=-4pt, scale=0.25, xdate=16pt, ydate=-10pt,
    date={\footnotesize01.01.2001}]
  }
  \projecttask{\projectsubtasknum}
  {Разработать архитектуру веб-сервиса}%
  {Текст ПЗ}%
  {27.04.2025}
  {\
    % % \signat[xsign=-10pt, ysign=4pt, scale=0.25, xdate=17pt, ydate=-2pt, date={\scriptsize01.01.2001}]
  }
  % ----------------------------------------
  \projecttask{\projectsubtasknum}
  {
    Выбрать стек технологий и
    описать причину его выбора
  }%
  {Текст ПЗ}%
  {01.05.2025}
  {
    % \signat[xsign=-12pt, ysign=-4pt, scale=0.25, xdate=14pt, ydate=-10pt, date={\footnotesize01.01.2001}]
  }
  % ----------------------------------------
  \projecttask{\projectsubtasknum}
  {Описать функционал веб-сервиса}%
  {Текст ПЗ}%
  {08.05.2025}
  {
    % % \signat[xsign=-10pt, ysign=4pt, scale=0.25, xdate=17pt, ydate=-2pt, date={\scriptsize01.01.2001}]
  }
  % ----------------------------------------
  \projecttask{\projectsubtasknum}
  {
    Описать программную реализацию системы
    (основные классы и функции)
  }%
  {Текст ПЗ}%
  {17.05.2025}
  {
    % \signat[xsign=-12pt, ysign=-4pt, scale=0.25, xdate=14pt, ydate=-10pt, date={\footnotesize01.01.2001}]
  }
  % ----------------------------------------
  % ----------------------------------------
  \projecttask{\bfseries\projecttasknum}{\bfseries Технологическая и практическая часть}{}{}{}
  % (указывается, что конкретно должно быть реализовано и протестировано, а также используемые для этого методы, инструментальные средства, технологии)
  \projecttask{\projectsubtasknum}
  {
    Протестировать систему на тренировочной
    базе данных
  }%
  {Исполняемые тестовые файлы, текст ПЗ}%
  {19.05.2025}
  {
    % \signat[xsign=-7pt, ysign=-7pt, scale=0.32, xdate=12pt, ydate=-17pt, date={01.01.2001}]
  }
  % ----------------------------------------
  \projecttask{\projectsubtasknum}
  {Описать процесс работы пользователя с веб-сервисом}%
  {Текст ПЗ}%
  {21.05.2025}
  {
    % % \signat[xsign=0pt, ysign=-5pt, scale=0.4, ydate=-20pt, date={\small01.01.2001}]
  }
  % ----------------------------------------
  \projecttask{\bfseries\projecttasknum}
  {\itshape
    Оформить пояснительную записку (ПЗ) и
    иллюстративный материал для доклада
  }%
  {Текст ПЗ, презентация}%
  {22.05.2025}
  {
    % \signat[xsign=-12pt, ysign=-4pt, scale=0.25, xdate=14pt, ydate=-10pt, date={\footnotesize01.01.2001}]
  }
  % ----------------------------------------
  % ----------------------------------------
}

\taskliterature{
  \section{== TODO ==}
  \nocite{
    sidorovEstestvennyeIskusstvennyeYazyki2024,
    kimNaturalLanguageSQL2020,
    borodinZadacheSostavleniyaZaprosov2016,
    polonskiyPerevodRusskogoEstestvennogo2023,
    LLMyPreobrazovaniiZaprosa2023,
    zhuAutomatedCrossdomainExploratory2024
  }
}

% # Подписи

% Для простановки подписи используются слдеющие команды:
% - простая подпись: \sign[<сдвиг>]{<масштаб>}{<FIO>}
% где 
% - <сдвиг> --- необязательный сдвиг подписи по вертикали для правильного
%   расположения относительно строки
% - <масштаб> --- число, используемое для масштарибования изображения подписи
% - <FIO> --- имя файла с подписью, файл должен быть помещен в
%   img/signatures/FIO.png и иметь прозрачный фон
% - <дата> --- дата, которая будет выведена под подписью
%
% - подпись с датой: \signat[xsign=, ysign=, scale=, img=, xdate=, ydate=, date=]
% где все аргументы необязательны и имеют дефолтные значения:
% - xsign(=-10pt) --- сдвиг картинки подписи по горизонтали
% - ysign(=-4pt) --- сдвиг картинки подписи по вертикали
% - scale(=0.3) --- число, используемое для масштарибования изображения подписи
% - img(=supervisor) --- название изображения подписи в папке local или public
% - xdate(=10pt) --- сдвиг даты по горизонтали
% - ydate(=-10pt) --- сдвиг даты по вертикали
% - date(=01.01.2001) --- дата


% ## Утверждение задания руководителем и студентом
\authortaskapproval{\sign[5pt]{.15}{author}}
\supervisortaskapproval{\sign[0pt]{.35}{supervisor}}

% ## Утверждение РСПЗ руководителем, студентом и консультантом
\authorrspzapproval{\sign[10pt]{.20}{author}}
\supervisorrspzapproval{\sign[0pt]{.35}{supervisor}}
\consultantrspzapproval{}

% ## Оценка руководителя и консультанта за РСПЗ
\supervisorrspzgrade{10 б.}
\consultantrspzgrade{}

% ## Утверждение ПЗ руководителем, студентом и консультантом
\authorpzapproval{\sign[10pt]{.20}{author}}
\supervisorpzapproval{\sign[0pt]{.35}{supervisor}}
\consultantpzapproval{}

% ## Оценка руководителя за ПЗ
\supervisorpzgrade{12 б.}
\consultantpzgrade{}
