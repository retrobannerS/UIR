\chapter*{Реферат}
\thispagestyle{plain}

Общий объем основного текста, без учета приложений~---
\pageref{end_of_main_text} страниц, с учетом приложений~---
\pageref{end_of_document}. Количество использованных источников~---
\hyperref[sec:bibliography]{\total{citenum}}. Количество приложений~--- 
\hyperref[sec:appendices]{\total{totalappendices}}.

%Ключевые слова: 
\noindent \uppercase{ключевое слово 1, ключевое слово 2, \dots .}

Целью данной работы является \dots

В первой главе проводится обзор и анализ \dots

Во второй главе описываются использованные и разработанные/модифицированные методы/модели/алгоритмы \dots.

В третьей главе приводится описание программной реализации и экспериментальной проверки \dots.

В приложении \ref{app-format} описаны основные требования к форматированию пояснительных записок к дипломам и (магистерским) диссертациям.

В приложении \ref{app-structure} представлена общая структура пояснительной записки.

В приложении \ref{app-manual} приведены некоторые дополнительные комментарии к использованию данного шаблона.
