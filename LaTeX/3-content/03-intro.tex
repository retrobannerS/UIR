\chapter*{Введение}
\label{sec:afterwords}
\addcontentsline{toc}{chapter}{Введение}

Базы данных (БД) являются неотъемлемой
частью множества сфер деятельности, включая научные исследования, бизнес,
государственное управление и многие другие. Они предоставляют быстрый и
эффективный доступ к огромным массивам данных~\cite{savoskinIssledovanieSposobovPrimeneniya2019}.

Для экспертов в доменной области,
обладающих знаниями языков запросов, получение данных является удобным и простым процессом.
Однако для пользователей, не являющихся экспертами, особенно тех, кто работает
вне IT-сферы, извлечение информации из БД предствляет значительную трудность.
Она представляет из себя барьер: необходимость владения специальными навыками,
в частности, составлению запросов с помощью специальных языков.
Ситуация усугубляется для тех, чей родной язык отличается от
английского. Поскольку в большинстве случаев формальные языки взаимодействия с базами
данных являются англоязычными, возникает дополнительный языковой барьер, который
еще больше усложняет взаимодействие
ними~\cite{karimiNaturalLanguageQuery2022,bolyabkinIntellektualnayaSistemaDlya2021}.

Таким образом, формируется категория пользователей, которые имеют доступ к
необработанным данным и испытывают потребность в их анализе, но не могут самостоятельно
извлечь информацию из-за отсутствия технических навыков работы с базами данных
и/или языкового барьера.

Существующие пользовательские интерфейсы, адаптированные для данной категории,
могут быть «менее дружелюбными или вовсе отсутствовать»,
что создает потребность в альтернативных решениях~\cite{karimiNaturalLanguageQuery2022}.
Они позволят сократить время при принятии решений, ускорят бизнес-процессы в компаниях,
а также позволят их командам быстрее обучаться работе с данными и сократит нагрузку на дата-аналитиков.
Более того, такие системы могут стать полезны для людей,
нашедших актуальную информацию в открытом доступе и проводящих исследования в совершенно другой
области наук, как правило, далёкой от технической.
Также они могут стать полезны для людей с ограниченными возможностями или
пожилых пользователей, которым может быть физически трудно взаимодействовать с компьютером
традиционным способом.

История создания данных интерфейсов берет начало еще в 1973 году, когда была спроектирована
система LUNAR была применена для ответа на вопросы о лунных породах, доставленных прямиком 
с естественного спутника Земли~\cite{zhuLargeLanguageModel2024}. Основная загвоздка при создании данных 
систем лежит в решении задачи перевода естественного языка на формальный язык запросов SQL.
Количество исследований, фокусирующихся на решении данной задачи, резко возрасло, начиная
с июля 2024 года, что подтверждает научный интерес к данной 
теме~\cite{huangExploringLandscapeTexttoSQL2025}.

Таким образом, интерфейсы, позволяющие простым пользователям компьютера производить общение
с базами данных, призваны демократизировать доступ к данным и сделать их доступными
для более широкого круга пользователей, которые в ней нуждаются, независимо от их технической подготовки.

Целью данной работы является анализ современных подходов к упрощению взаимодействия пользователей с 
базами данных и, на основе этого анализа, разработка и тестирование прототипа веб-интерфейса, 
позволяющего выполнять запросы к ним с использованием естественного языка.