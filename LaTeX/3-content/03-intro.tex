\chapter*{Введение}
\label{sec:afterwords}
\addcontentsline{toc}{chapter}{Введение}

Базы данных (БД) являются неотъемлемой
частью множества сфер деятельности, включая научные исследования, бизнес,
государственное управление и многие другие. Они предоставляют быстрый и
эффективный доступ к огромным массивам данных~\cite{savoskinIssledovanieSposobovPrimeneniya2019}.

Для экспертов в доменной области,
обладающих знаниями языков запросов, получение данных является удобным и простым процессом.
Однако для пользователей, не являющихся экспертами, особенно тех, кто работает
вне IT-сферы, извлечение информации из БД предствляет значительную трудность.
Она представляет из себя барьер: необходимость владения специальными навыками,
в частности, составлению запросов с помощью специальных языков.
Ситуация усугубляется для тех, чей родной язык отличается от
английского. Поскольку в большинстве случаев формальные языки взаимодействия с базами
данных являются англоязычными, возникает дополнительный языковой барьер, который
еще больше усложняет взаимодействие
ними~\cite{karimiNaturalLanguageQuery2022,bolyabkinIntellektualnayaSistemaDlya2021}.

Таким образом, формируется категория пользователей, которые имеют доступ к
необработанным данным и испытывают потребность в их анализе, но не могут самостоятельно
извлечь информацию из-за отсутствия технических навыков работы с базами данных
и/или языкового барьера.

Существующие пользовательские интерфейсы, адаптированные для данной категории,
могут быть «менее дружелюбными или вовсе отсутствовать»,
что создает потребность в альтернативных решениях~\cite{karimiNaturalLanguageQuery2022}.
Они позволят сократить время при принятии решений, ускорят бизнес-процессы в компаниях,
а также позволят их командам быстрее обучаться работе с данными и сократит нагрузку на дата-аналитиков,
демократизируя доступ к данным и делая их доступными для более широкого круга пользователей.

Целью данной работы является проектирование модели интеграции современного
многоагентного NLIDB-ядра XiYan-SQL во внешний веб-сервис посредством
разработки модели компонента-оркестратора (MCP-клиента).
Практическим подтверждением жизнеспособности предложенной модели служит ее
реализация в виде программного прототипа, предоставляющего
естественно-языковой интерфейс к базам данных.

В данной работе для решения обозначенной проблемы проводится комплексный анализ
современных подходов. В первом разделе рассматривается эволюция пользовательских
интерфейсов к базам данных, обосновывается выбор в пользу естественно-языковых
решений (NLIDB) и исследуется история развития технологии Text-to-SQL.
На основе этого анализа для реализации прототипа выбирается и обосновывается передовое
технологическое ядро~--- открытый фреймворк XiYan-SQL, лидирующий в
ключевых отраслевых бенчмарках.

Далее, работа переходит от выбора технологии к моделированию взаимодействия
серверной части веб-сервиса с XiYan-SQL для интеграции.
Во втором разделе описывается современная многоагентная архитектура, лежащая
в основе XiYan-SQL, и протокол взаимодействия \textit{Model Context Protocol (MCP)},
который является стандартом для таких систем. Центральным элементом раздела
становится проектирование архитектуры \textbf{MCP-клиента}~--- компонента-оркестратора,
который будет управлять сложным жизненным циклом обработки запроса на стороне
разрабатываемого веб-сервиса. В заключительном разделе эта теоретическая модель интеграции
транслируется в инженерное решение: с помощью UML-диаграмм
проектируется архитектура прототипа веб-сервиса, описывается
технологический стек для его реализации, приводится программная реализация и тестирование.

Таким образом, ключевым результатом работы является проект архитектурного
решения для встраивания современных многомодульных Text-to-SQL систем
в прикладные приложения, а также его прототипная реализация с эмуляцией
внешнего компонента, что закладывает фундамент для дальнейшей
полноценной разработки.